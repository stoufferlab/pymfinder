\documentclass[12pt]{article} 
\usepackage{amsmath} 
\usepackage[dvips]{graphicx}
\usepackage{multirow} 
\usepackage{geometry} 
\usepackage{pdflscape}
\usepackage[labelfont=bf]{caption} 
\usepackage{setspace}
\usepackage[running]{lineno} 
\usepackage[round]{natbib} 
\usepackage{array}

\newcommand{\methods}{\textit{Materials \& Methods}}
\newcommand{\SI}{\textit{Appendix}~}

\topmargin -1.5cm % 0.0cm 
\oddsidemargin 0.0cm % 0.2cm 
\textwidth 6.5in
\textheight 9.0in % 21cm
\footskip 1.0cm % 1.0cm

\usepackage{authblk}

\title{\emph{pymfinder}: Tool Guide}

\author{Bernat Bramon Mora$^{1}$, Alyssa R. Cirtwill$^{2}$, Daniel B. Stouffer $^{1}$} 
\date{\small
$^1$Centre for Integrative Ecology, School of Biological Sciences, University of Canterbury, Christchurch, New Zealand\\
\medskip
$^2$Department of Physics, Chemistry, and Biology (IFM), Link\"{o}ping University, Link\"{o}ping, Sweden\\ }

\renewcommand\Authands{ and }

\begin{document} 
\maketitle 
\raggedright
\setlength{\parindent}{15pt} 

\newpage



\section{General information}

	\textbf{Description:} \textit{pymfinder} is a Python package designed to detect motifs in complex networks and define the roles of nodes and links using these motifs. Unipartite networks can be analysed using three-species motifs and bipartite networks can be analysed using motifs containin up to six species. Both weighted and binary networks can be analyzed. At its core, \emph{pymfinder} is a combination of Python methods for network-motif analysis as well as a Python wrapper for the original mfinder version 1.2 written in C and available at \emph{http://www.weizmann.ac.il/mcb/UriAlon/}~\citep{Alonwebsite}. This code has been included and modified here with the explicit consent of Nadav Kashtan, the author of mfinder 1.2.

	\textbf{Copyrights:} TBD

	\textbf{Version info:} TBD

	\textbf{Availability:} TBD

	\textbf{Platforms:} Windows, Linux, Mac OSX. Following recommendations for mfinder, large and dense networks (\textgreater10 000 nodes) require a computer with at least 512 Mbyte RAM in order to calculate motif frequencies. Calculating node or link roles will require greater resources.


\section{How to use \emph{pymfinder}}

	\subsection{Download and installation}

		\subsubsection{Download}

			The \emph{pymfinder} package can be downloaded from Github (url here) or [[journal?]].

		\subsubsection{Installation}

			Installation within a command-line terminal should be straightforward using the function 'setup.py' included in the \emph{pymfinder} package. After navigating to the directory containing the package, run:

			\textbf{python setup.py install}

			If an error message of 'Permission denied' or similar is returned, run:

			\textbf{python setup.py install --user}

			This will install \emph{pymfinder} locally rather than in the global Python site-packages or dist-packages directory. 

			% If the mac is being stupid, can install with python setup.py install --user --prefix=""

		\subsubsection{Checking installation}

			After installation, running the included test suite is strongly encouraged. This may be accomplished by running:

			\textbf{python setup.py test}

			% Does not work on Mac, going to test on James' linux build.
			% Seems to be some form of XCode error


	\subsection{Basic useage}


		\subsection{Input file format}

			Input network file format should be in simple '.txt' format. Species names may be given as text or integers but should \textbf{not} include spaces. Each edge should be represented by a line of the following format:

			\textbf{\textless source node\textgreater  \textless target node\textgreater}

			Examples:\\
			1 2\\
			3 1\\
			Salmo\_trutta midge\\
			Corvus\_corax Salmo\_trutta\\


			If interaction strengths are known, they can be passed to \emph{pymfinder} in the input file. In this case, each edge should be represented by a line with the format:

			\textbf{\textless source node\textgreater \textless target node\textgreater \textless interaction strength\textgreater}

			Examples:\\
			1 2 1\\
			3 1 2.5\\
			Salmo\_trutta midge 0.005\\
			Corvus\_corax Salmo\_trutta 3\\

		\subsubsection{Function call and arguments}

			All of the functions within \emph{pymfinder} can be called using the same framework. Within a Python environment, first import the \emph{pymfinder} package using, for example:

			\textbf{import pymfinder as py}

			The motif structure, motif participation, and motif roles for the network can then be calculated simultaneously using:

			\textbf{results = py.pymfinder(network,\\
								links=False,\\
	              motifsize = 3,\\
	              stoufferIDs = None,\\
	              allmotifs = False,\\
	              nrandomizations = 0,\\
	              randomize = False,\\
	              usemetropolis = False,\\
	              networktype = "unipartite")}.\\


			The pymfinder function call includes the following arguments:

     	\begin{itemize}

     		\item \textbf{network}: Network path. Must be in quotation marks. No default given.

				\item \textbf{links}: Determines whether or not to calculate statistics for links as well as nodes. If \textbf{links = True}, link participation and roles will be calculated. Defaults to \textbf{links = False}.

				\item \textbf{motifsize}: Size of motifs to be calculated. Defined for motifsize=3 for unipartite networks and 3$\leq$motifsize$\leq$6 for bipartite networks. Defaults to \textbf{motifsize = 3}.

				\item \textbf{stoufferIDs}: Determines whether to label motifs following~\citet{StoufferIDpaper} or~\citet{Numberpaper}. If \textbf{stoufferIDs = True}, labels will be as in~\citet{StoufferIDpaper}. Defaults to \textbf{stoufferIDs = False}.

				\item \textbf{allmotifs}: Not sure what this does. Defaults to \textbf{allmotifs = False}.

				\item \textbf{nrandomizations}: Number of random networks with which to compare the observed network. Defaults to 0 (no randomizations performed).

				\item \textbf{randomize}: Determines whether to conduct the randomizaiton analysis. If false, no randomizations will be conducted regardless of the value given to \textbf{nrandomizations}. Defaults to \textbf{randomize = False}.

				\item \textbf{usemetropolis}: If randomizations are to be performed, determines whether to use the Metropolis algorithm. If Metropolis is not used, \emph{pymfinder} uses an MCMC algorithm to shuffle the original network while preserving in- and out-degrees of nodes. Defaults to \textbf{usemetropolis=False} (MCMC-based randomizations).

				\item \textbf{networktype}: Indicates whether the network is unipartite (all species may interact with all other species) or bipartite (species are divided into two groups and may interact between groups but not within a group). Defaults to \textbf{networktype = unipartite}.

			\end{itemize}

		\subsubsection{Output}

			The object 'results' returned by \emph{pymfinder} is a NetworkStats object containing dictionaries of motifs, nodes, and links as well as the network type (uniparite or bipartite) and the size of motifs used. The value for each motif in the .motifs dictionary is an Motif object containing the motif profile for that motif. Similarly, the value for each node or link in the .nodes or .links dictionaries is a NodeLink object containing the motif participation or role of that node or link.


			These results can be collected into text-formatted tables and may be seen using:

			\textbf{print results}

			or written to a file using:

			\textbf{f=open('filename','w')\\
					f.write(str(results))\\
					f.close()}\\
			\medskip


			If \textbf{links=False}, the results include three tables. The first presents the motif profile of the network. Each row gives a motif ID, the count of that motif in the observed network, the mean and standard deviation of the count of that motif in the randomized netowrks, and the $Z$-score comparing the real network to the randomized networks. If \textbf{ranodmizations=False}, the random mean and standard deviations will be reported as 0.000 and the $Z$-score will be given as 888888.000.



			The second table presents the motif participation for each node in the network. Each row gives a node ID and the number of times the node appears in each motif. Motif ID's are given in the first row.


			The third table presents the role for each node in the network. Each row gives a node ID and the number of times the node appears in each position in each motif. Node positions are labelled using the following notation: (motif ID, number of predators/out links, number of prey/in links).


			If \textbf{links=True}, the results will also contain tables presenting links' motif participation and roles. Links' motif participation follows nodes' motif participation and link roles follow node roles. In both cases, the format of the output table echoes that of the node tables. The only notable difference is in the labelling of link positions. Rather than (motif ID, number of out links, number of in links), link positions are labelled (motif ID, (out links for node 1, in links for node 1), (out links for node 2, in links for node 2)) where nodes 1 and 2 are the two species connected by link (1,2).


			[[Would examples of these tables be helpful to include as figures?]]

\section{Algorithms}

	[[I think the manuscript describes the algorithms rather well. Do we actually need this?]]

\section{Figures}


	[[Which ones do we want? I assume the three-species unipartite and 2-6 species bipartite, with positions labelled. Any others? Maybe one giving the StoufferIDs and numerical IDs for three-species motifs.]]


\end{document}

